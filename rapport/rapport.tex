\documentclass[a4paper,10pt]{article} %type de document et paramètres


\usepackage{lmodern} %police de caractère
\usepackage[english,french]{babel} %package de langues
\usepackage[utf8]{inputenc} %package fondamental
\usepackage[T1]{fontenc} %package fondamental

\usepackage[top=3cm, bottom=3cm, left=3cm, right=3cm]{geometry} %permet de

\usepackage[pdftex, pdfauthor={Maxime Flin, Pierre Gimalac}, pdftitle={Rapport projet}, pdfsubject={chatgroup with udp flooding}, pdfkeywords={UDP, chatgroup, flooding}, colorlinks=true, linkcolor=black]{hyperref}

\author{Maxime Flin \& Pierre Gimalac}
\title{Rapport de Projet}

\begin{document}
\maketitle

\section{Compilation et exécution}
Le projet se compile grâce au makefile fourni, en faisant la commande \textit{'make all'}.\\

L'exécutable produit s'appelle \textit{chat}. Il possèdes des arguments optionnels qui sont dans l'ordre:
\begin{enumerate}
\item un numéro de port (un numéro aléatoire sera choisi si l'argument n'est pas fourni)
\item ``0'', ``1'', ``STDOUT'', ``STDERR'', ou un chemin vers un fichier (potentiellement non existant), indiquant où doivent être écris les logs. Par défaut les logs sont écris sur STDERR.
\item un pseudonyme (un pseudonyme aléatoire sera choisi dans une liste si l'argument n'est pas fourni).
\end{enumerate}

\subsection{Commandes}
L'interface possède un certain nombre de commandes, qui commencent par /:
\begin{description}
\item[add <addr> <port>] ajoute aux voisins potentiels les adresses associées à <addr> et <port>.
\item[name <name>] change le pseudonyme en <name>.
\item[random] change le pseudonyme de manière aléatoire.
\item[print] affiche la liste des voisins et voisins potentiels.
\item[juliusz] équivalent à ``/add jch.irif.fr 1212''
\item[neighbour] force l'envoi de tlv neighbour.
\item[clear] efface le terminal (fonctionne au moins sur bash et zsh).
\item[chid] change l'id de manière aléatoire.
\item[transfert <path>] envoie le fichier <path> avec le protocole de fragmentation.
\item[help] affiche l'ensemble des commandes.
\item[quit] quitte programme, envoie des tlv goaway au passage.
\end{description}

\section{Interface}
\section{Sujet minimal}
\section{Extensions}
\subsection{Tlv 220}
\section{Extensions considérées mais non implémentées}
\section{Détails d'implémentation}



\end{document}

 % - introduction : projet fait par machin et bidule.
 %  - manuel
 %     - comment compiler
 %     - comment s'en servir
 %  - partie qui marche
 %     - sujet minimal
 %     - extensions
 %  - partie qui ne marche pas
 %     - qui ne marche pas parce qu'on est incompétents
 %     - qui ne marche pas parce qu'on a eu la flemme
 %  - parties pompées sur les copains (à qui merci)
 %  - détails d'implémentation
 %     - structure générale du programme
 %     - parties non-triviales que je vous explique gentiment
 %     - parties super bien faites dont on est fiers
 %     - parties mal faites dont on a un peu honte mais qu'on a la flemme
 %       d'améliorer
 %  - commentaires sur le protocole
 %     - telle partie du protocole est mal faite, voici comment l'améliorer
 %     - telle partie du protocole est difficile à implémenter, je vous déteste
 %  - conclusion
 %     - ce sujet nous a énormément apporté, il nous a ouvert l'esprit et
 %       a amélioré nos performances sexuelles.
