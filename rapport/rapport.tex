\documentclass[a4paper,10pt]{article} %type de document et paramètres


\usepackage{lmodern} %police de caractère
\usepackage[english,french]{babel} %package de langues
\usepackage[utf8]{inputenc} %package fondamental
\usepackage[T1]{fontenc} %package fondamental

\usepackage[top=3cm, bottom=3cm, left=3cm, right=3cm]{geometry} %permet de

\usepackage[pdftex, pdfauthor={Maxime Flin, Pierre Gimalac}, pdftitle={Rapport projet}, pdfsubject={chatgroup with udp flooding}, pdfkeywords={UDP, chatgroup, flooding}, colorlinks=true, linkcolor=black]{hyperref}


\usepackage{url} %permet de mettre des URL actifs \url{}
\let\urlorig\url
\renewcommand{\url}[1]{\begin{otherlanguage}{english}\urlorig{#1}\end{otherlanguage}}

\author{Maxime Flin \& Pierre Gimalac}
\title{Rapport de Projet}

\begin{document}
\maketitle

\begin{abstract}
  Ce rapport est un compte rendu du travail fourni pour le projet du cours de programmation réseau. Nous avons implémenté un chat décentralisé selon le protocole d'inondation décrit par Juliusz Chroboczek dans le sujet et \textbf{interopérable avec son implémentation}.

  Plusieurs extensions ont été développées parmi lesquelles \textbf{l'agrégation de tlv(\ref{sec:agrega})}, \textbf{la fragmentation de gros messages(\ref{sec:frag})}, \textbf{quelques optimisations de l’inondation}, \textbf{l'implémentation d'une interface web(\ref{sec:web})}, \textbf{la découverte du PMTU(\ref{sec:pmtu})} et de la \textbf{concurrence(\ref{sec:con})}.
\end{abstract}

\section{Compilation et exécution\label{sec:exec}}
L'application est développé en C pour les systèmes UNIX respectant les normes POSIX. Elle utilise l'API socket, les fonctions de la \textrm{glibc} et la fonction \textrm{SHA1} de la bibliothèque \textrm{OpenSSL}.

Un \textrm{Makefile} est fournis pour compiler le projet:  \textit{'make all'}.\\

L'exécutable produit s'appelle \textit{chat}.
\begin{itemize}
\item \textrm{-{}-port} ou \textrm{-p} suivi du numéro de port définit le port sur lequel lancer le serveur de chat. Si aucun port n'est précisé, il est alors choisit aléatoirement.
\item \textrm{-{}-webport} ou \textrm{-w} suivi du port sur lequel faire lancer le serveur web de l'application. Si aucun port n'est précisé, alors il est choisit aléatoirement entre 8080 et 8089.
\item \textrm{-{}-pseudo} suivit d'un pseudonyme. Un pseudonyme aléatoire est choisi dans une liste s'il n'est pas précisé.
\item \textrm{-{}-logs} ou \textrm{-l} suivit du chemin relatif vers un fichier où écrire les logs. Si le fichier n'existe pas il est créé. Si aucun fichier de log n'est précisé, les logs de l'application sont affichés sur la sortie standard d'erreur.
\end{itemize}

La commande suivante lance donc un serveur de chat sur le port 1212 accessible depuis une interface web sur le port 8089 (\href{localhost:8089}{localhost:8089}) et l'utilisateur de interface dans le terminal a pour pseudonyme Bob.

\begin{verbatim}
./chat --port 1212 --web-port 8089 --pseudo Bob
\end{verbatim}

Aucune option n'est nécessaire pour lancer le programme.

\section{Interface}

L'interface dans le terminal est en ligne de commande, elle interagit donc en lisant les commandes écrites sur l'entrée standard. \textbf{Une commande commence toujours par /}. Les commandes que propose notre application sont les suivantes:
\begin{description}
\item[add <addr> <port>] ajoute aux voisins potentiels les adresses associées à \textit{addr} et \textit{port}.
\item[name <name>] change le pseudonyme en \textit{name}.
\item[random] change le pseudonyme de manière aléatoire.
\item[print] affiche la liste des voisins et voisins potentiels.
\item[juliusz] équivalent à ``/add jch.irif.fr 1212''.
\item[neighbour] force l'envoi de tlv neighbour.
\item[clear] efface le terminal (fonctionne au moins sur bash et zsh).
\item[chid] change l'identifiant de manière aléatoire.
\item[transfert <type> <path>] envoie le fichier \textit{path} avec le protocole de fragmentation. Quel que soit le fichier, il est envoyé avec le type donné. La description des types est données dans la section \ref{sec:frag}.
\item[switchlog [<logfile>]] si un fichier est donné en argument, la sortie log s'écrit sur le fichier donné (qui pour des raisons de sécurité ne doit pas exister et sera créé). S'il n'y a pas d'argument, alors si les logs étaient activés ils sont éteints et vice-versa (alors écrits sur stderr).
\item[help] affiche l'ensemble des commandes.
\item[quit] quitte programme, envoie des tlv goaway au passage.
\end{description}

Tout ce qui n'est pas une commande, ie. tout ce qui ne commence pas par un \textrm{/}, est un message qui sera envoyé sur le réseau. Le préfixe \textit{pseudo: } sera automatiquement ajouté devant le message.\\

Nous affichons par défaut (beaucoup) de logs sur la sortie d'erreur (comme dit dans la section \ref{sec:exec}, ils peuvent être redirigé dans un fichier avec l'option \textbf{logs}, ou arrêtés avec \textbf{switchlog}). Pour y voir plus clair, nous avons ajouté un code couleur au différents type de messages\footnote{L'affichage peut être totalement personnalisé dans \textrm{interface.h}: couleurs de police et de fond, italique, gras, texte clignotant,...}. Nous avons choisi d'afficher la sortie standard dans la couleur normale, les logs en jaune et les erreurs en rouge. Pour ce faire nous avons implémenté la fonction \textit{cprint} dans \textit{utils} qui étends \textit{dprintf} en mettant les messages à la bonne couleur selon la sortie voulue.\\

Pour ajouter en voisin l'interface de Juliusz et envoyer les deux messages \textit{Bob: Bonjours tout le monde !} et \textit{Bob: Je suis très heureux de vous voir.}, il faudra donc entrer les commandes suivantes.

\begin{verbatim}
/add jch.irif.fr 1212
Bonjour tout le monde !
Je suis très heureux de vous voir.
\end{verbatim}

Une interface web a aussi été implémentée, mais étant une extension nous en reparlerons dans la section \ref{sec:web}.

\section{Fonctionnalitées}

\subsection{Sujet minimal}
Le sujet minimal a été entièrement implémenté.
Nous avons ajouté un gestionnaire de signal lors de l'interruption du programme pour envoyer un TLV GoAway à tous les voisins indiquant que nous quittons le réseau.

Nous inondons les data à des temps aléatoires pour augmenter la probabilité de réduire le nombre d’inondation à faire d'un même message.

Nous avons soumis notre implémentation à des séries de tests implacables pour tester son efficacité et sa résistance aux paquets buggés.\\


\textit{Les extentions suivantes sont présentées dans l'ordre dans lequel elles ont été implémentées dans le projet.}

\subsection{Agrégation de TLV\label{sec:agrega}}
Les messages a envoyer sont stocké dans une file (plus de détails dans la section \ref{sec:struct}) de messages accessible grâce à deux méthodes:

\begin{description}
\item[pull\_tlv] Retire un message à envoyer de la file des messages à envoyer.
\item[push\_tlv] Cette fonction attends un TLV et un voisin en paramètre. Elle ajoute le tlv donné à un message si avec cet ajout la taille du message reste plus petite que le PMTU de ce voisin. Si un tlv ne peut être inséré dans aucun message, un nouveau message est ajouté à la fin de la file.
\end{description}

Chaque voisin possède un attribut \textrm{PMTU} correspondant à une approximation inférieure du réel \textrm{PMTU} avec ce voisin. Nous avons plus tard implémenté la découverte de PMTU (section \ref{sec:pmtu}).


\subsection{Fragmentation des gros messages\label{sec:frag}}
Nous avons implémenté le protocole de fragmentation de gros messages proposé par \textrm{Alexandre Moine} et \textrm{Tristan François} sur la mailing liste du projet. Ils proposaient d'utiliser un nouveau type de data afin que les TLV puissent être inondés même par les pairs n'ayant pas implémenté la fragmentation et permettre à ceux qui l'avaient faite de reconstituer le message à l'arrivée.

Le format des TLV est le suivant:

\begin{verbatim}
         0                   1                   2                   3
         0 1 2 3 4 5 6 7 8 9 0 1 2 3 4 5 6 7 8 9 0 1 2 3 4 5 6 7 8 9 0 1
        +-+-+-+-+-+-+-+-+-+-+-+-+-+-+-+-+-+-+-+-+-+-+-+-+-+-+-+-+-+-+-+-+
        |    Type = 4   |   Length      |                               |
        +-+-+-+-+-+-+-+-+-+-+-+-+-+-+-+-+                               +
        |                            Sender Id                          |
        +                               +-+-+-+-+-+-+-+-+-+-+-+-+-+-+-+-+
        |                               |             Nonce             |
        +-+-+-+-+-+-+-+-+-+-+-+-+-+-+-+-+-+-+-+-+-+-+-+-+-+-+-+-+-+-+-+-+
        |                               |   Type = 220  |               |
        +-+-+-+-+-+-+-+-+-+-+-+-+-+-+-+-+-+-+-+-+-+-+-+-+-+-+-+-+-+-+-+-+
        |           Nonce du message fragmenté          |     Type      |
        +-+-+-+-+-+-+-+-+-+-+-+-+-+-+-+-+-+-+-+-+-+-+-+-+-+-+-+-+-+-+-+-+
        |    Taille totale du message   |      Position du fragment     |
        +-+-+-+-+-+-+-+-+-+-+-+-+-+-+-+-+-+-+-+-+-+-+-+-+-+-+-+-+-+-+-+-+
        |                      Fragment du message...
        +-+-+-+-+-+-+-+-+-+-+-+-+-+-+-+-+-+-+-+-+-+-+-+-+-+-+-+-+-+
\end{verbatim}

On reconnaît la structure d'un TLV Data auquel on a ajouté les champs \textit{Nonce du message fragmenté}, \textit{Type}, \textit{Taille totale du message}, \textit{Position du fragment} et \textit{Fragment}. Aussi le message est correctement inondé par les pairs qui n'ont pas implémenté le protocole, le \textit{Nonce} du data étant différent pour chaque message. Les fragments d'un même message peuvent être rassemblés grâce au champ \textit{Nonce du message fragmenté} partagé par tous les fragments de ce dernier.

Le champ type permet de savoir comment interpréter le message reçu. Au cours de nos échanges sur la mailing liste, nous avons proposé les attributions suivantes qui ont été acceptés.

\begin{description}
\item[0] Texte UTF-8
\item[2] Image gif
\item[3] Image jpg
\item[4] Image png
\item[5] SVG
\end{description}

Cette attribution de type nous permet notamment de différentier la réception d'un long texte de la réception d'une image. Si nous recevons une image nous pouvons alors l'afficher dans notre interface web.

Nous rassemblons tous les messages fragmentés que nous recevons, une analyse du type pourrait être faite pour ne pas s'encombrer de gros messages dont ne feront rien une fois rassemblés.

Des tests ont été fait avec \textrm{Alexandre Moine} et \textrm{Tristan François} pour s'assurer de l'interopérabilité de nos implémentations.

\subsection{Interface web, protocole HTTP et WebSocket\label{sec:web}}

Après avoir jeté un coup d’œil à l'implémentation de l'interface web de Julius, nous avons remarqué qu'il utilisait le protocole WebSocket pour communiquer entre son interface web (\href{http://jch.irif.fr:8082/}{http://jch.irif.fr:8082/}) et son application, accessible sur le port 1212. Nous aimions surtout l'idée d'implémenter une interface graphique, non pas grâce à l'utilisation d'une bibliothèque graphique lourde et compliquée, mais à travers l'implémentation d'un protocole réseau. De plus, le besoin d'une interface graphique se faisait ressentir à cause de la quantité astronomique de logs que nous affichions, il nous a donc semblé que cette solution permettait de le faire tout en restant dans les bornes de ce cours.\\

Nous avons donc implémenté un serveur \textrm{HTTP}\footnote{\href{https://tools.ietf.org/html/rfc2616}{RFC2616}} très basique pour afficher une page web, créer la connexion WebSocket et charger des images depuis un dossier temporaire. Il n'est pas très sécurisé et n'implémente pas tous les détails du protocole, nous sommes au courant, mais l'objectif étant d'avoir une interface graphique simple dans le navigateur et surtout d'implémenter le protocole WebSocket, nous avons mis de côté ces points là.\\

Nous avons ensuite implémenté le protocole \textrm{WebSocket}\footnote{\href{https://tools.ietf.org/html/rfc6455}{RFC6455}}. La connexion WebSocket s'effectue après un ``\textit{Handshake}'' à travers une connexion \textrm{HTTP} ayant le champs \textrm{Upgrade: websocket}. On reçoit une clef secrète \textrm{Sec-WebSocket-Key}, on la concatène à une ``\textit{magic string}'', on la hash avec \textrm{SHA1} puis on envoie une réponse \textrm{HTTP} de type 101 contenant le champ \textrm{Sec-WebSocket-Accept} ayant pour valeur l'encodage en base64 du hash. À partir de ce moment la connexion TCP n'est plus fermée avant que l'un des deux pairs ne le décide.
Une trame WebSocket est au format suivant

\begin{verbatim}
          0                   1                   2                   3
          0 1 2 3 4 5 6 7 8 9 0 1 2 3 4 5 6 7 8 9 0 1 2 3 4 5 6 7 8 9 0 1
         +-+-+-+-+-------+-+-------------+-------------------------------+
         |F|R|R|R| opcode|M| Payload len |    Extended payload length    |
         |I|S|S|S|  (4)  |A|     (7)     |             (16/64)           |
         |N|V|V|V|       |S|             |   (if payload len==126/127)   |
         | |1|2|3|       |K|             |                               |
         +-+-+-+-+-------+-+-------------+ - - - - - - - - - - - - - - - +
         |     Extended payload length continued, if payload len == 127  |
         + - - - - - - - - - - - - - - - +-------------------------------+
         |                               |Masking-key, if MASK set to 1  |
         +-------------------------------+-------------------------------+
         | Masking-key (continued)       |          Payload Data         |
         +-------------------------------- - - - - - - - - - - - - - - - +
         :                     Payload Data continued ...                :
         + - - - - - - - - - - - - - - - - - - - - - - - - - - - - - - - +
         |                     Payload Data continued ...                |
         +---------------------------------------------------------------+
\end{verbatim}

Le bit \textrm{FIN} est à 1 si sa charge utile (payload) est le dernier fragment de la séquence de fragments. \textrm{opcode} est un entier sur 4 bits qui encode le type du message.

\begin{description}
\item[0x0] La charge utile est la suite de celle de la trame précédente
\item[0x1] Texte
\item[0x2] Données binaires
  ...
\item[0x8] Fin de connexion
\item[0x9] Ping
\item[0xA] Pong
\end{description}

L'échange de ping et de pong permet de s'assurer que le pair reçoit encore les messages, la réception d'un ping doit être suivie*ù aussi vite que possible de l'envoi d'un pong. Notre implémentation du protocole n'envoie jamais de ping.

La communication entre le client web et notre serveur se fait à travers ce protocole.
\begin{enumerate}
\item Le client envoie du texte à travers la socket qui est interprétée comme celle de l'interface du terminal (sauf en ce qui concerne l'ajout du pseudo au début).
\item Le serveur envoie du texte au client qui l'affiche tel quel dans le navigateur. Cette partie est source de potentielles failles de sécurité (injection de JavaScript dans le client d'un autre), nous sommes au courant mais avons choisit d'ignorer le problème ayant d'autres choses à faire.
\end{enumerate}

\subsection{Découverte de PMTU\label{sec:pmtu}}

Pour pouvoir implémenter l'agrégation de messages, nous maintenons pour chaque voisin un entier non signé de 2 octets représentant le \textrm{PMTU}\footnote{Path Maximum Unit Transfert} avec celui-ci. Pour implémenter la découverte de \textrm{PMTU}, il nous suffisait donc de trouver un moyen de déterminer si un message de taille donnée était correctement reçu.

Nous avons ensuite remarqué que lorsqu'un TLV Data est reçu, il est toujours suivit d'un ACK en réponse. L'idée est donc de prendre régulièrement des Data et de les envoyer dans message que nous comblons avec des PADn afin d’accéder à la taille voulue. Si la donnée est reçue nous recevons un ACK, nous en déduisons donc que la taille du message est plus petite que le \textrm{PMTU}, si nous ne recevons pas de ACK nous supposons que nous sommes au dessus du \textrm{PMTU}. Ainsi nous pouvons faire une recherche dichotomique du PMTU.

On note que si le ACK se perds alors que le message est reçu, on se contente de baisser la borne supérieure du PMTU.

\subsection{Concurrence\label{sec:con}}
Nous avions une gestion de la concurrence sans parallélisation ; nous essayions de lire et d'envoyer un nombre de message raisonnable pour notre nombre de voisins symétriques. Plus tard nous avons implémenté un version plus efficace qui parallélise la lecture de l'entrée standard, la réception, l'envoi et l'interface web si possible.

Grâce à des tests particulièrement intensifs nous avons pu éliminer beaucoup de problèmes apportés par les threads (en particulier des deadlock), avec plus ou moins de facilité (la difficulté d'affichage d'une backtrace propre depuis le code C nous a un peu ralenti mais une fois certaines commandes de gdb connues, le problème disparaît).

La concurrence nous permet également d'améliorer grandement l'utilisation du terminal. L'utilisation de la bibliothèque readline nous a permis d'éviter qu'un message que l'on est en train de taper ne soit coupé par l'affichage des logs ou autre. On peut utiliser les chaînes de contrôles bash\footnote{\url{https://wiki.archlinux.org/index.php/Bash/Prompt\_customization\#Common\_capabilities}} pour effacer la ligne courante, afficher ce que l'on veut afficher puis grâce à readline on peut réafficher le texte précédemment tapé.

\subsection{Extensions envisagées mais non développées}

\begin{description}
\item[Très très gros messages] Ayant implémenté la fragmentation de gros messages nous nous réjouissions à l'idée de pouvoir s'envoyer des images. Nous nous sommes vite rendu compte que la taille moyenne des images aujourd'hui est plutôt de l'ordre du méga-octet. Nous avons alors proposé un type de Data similaire à celui des gros messages mais avec un taille totale sur 4 octets au lieu de 2. Toutefois, le nombre de TLV à inonder pour transférer des fichiers de cette taille est tellement grand que cet usage semble peu approprié à ce genre d'application.
\item[Sécurité cryptographique] Nous avions sérieusement réfléchi à une extension cryptographique et proposé à cet effet des TLV pour l'échange de clef sur la mailing-list. La complexité de l'implémentation d'une telle extension et l'arrivée des examens (ainsi que l'absence de réponse à notre dernier mail) nous ont forcé à laisser tomber.
\item[Multicast] Cette extension a été faite par d'autres groupes ; nous nous sommes demandé pourquoi ne pas la faire. Finalement, comme il est compliqué d'envoyer des messages en multicast au delà du lien local, nous avons trouvé que cette extension présentait peu d'intérêt et de ce fait n'avons pas trouvé la motivation de l'implémenter.
\end{description}

\section{Plus sur l'implémentation\label{sec:implem}}
\subsection{Organisation}

Les fichiers sources du projets sont dans le dossier \textrm{src}. Le code est répartit dans les fichiers selon l'ordre suivant:

\begin{description}
\item[types.h] contient les structures essentielles utilisées pour l'implémentation du protocole.
\item[main] contient la boucle principale du programme, une fonction d'initialisation et des fonctions de gestion de l'entrée standard.
\item[structs] contient les structures de données utilisées dans le projet: hashmap, hashset, liste chaînée, arraylist.
\item[interface] contient les fonctions liées à l'interface (changement de pseudonyme, gestion de commandes, affichages des messages).
\item[network] contient des fonctions générales liées au réseau.
\item[flooding, send, clean, tlv\_queue] contiennent les fonctions liées à l'inondation, à la réception et à l'envoi de messages.
\item[hello, neighbour] contiennent des fonctions liées aux voisins et à l'envoi de tlv neighbour.
\item[tlv, checkers, handlers, onsend] contiennent des fonctions de gestion de tlv (création, vérification et gestion à la réception,...).
\item[base64, webinterface] contiennent des fonctions liées à l'interface web.
\item[threads, signal] contiennent des fonctions liées aux threads et aux signaux.
\item[utils] contient des fonctions diverses utilisées tout au long du projet.
\end{description}

\subsection{Structures utilisées\label{sec:struct}}
Les voisins sont stockés dans des hashset. \textrm{neighbours} contient les voisins symétriques et \textrm{potential\_neighbours} les voisins potentiels.

Les messages reçus récemment et les messages à inonder sont chacun stockés par une hashmap, respectivement \textrm{data\_map} et \textrm{flooding\_map}. Dans le second cas, il s'agit d'une hashmap de hashmap. À un message on associe un map qui fait la correspondance entre un voisin encore à inonder et les informations relative à cette inondation (date du dernier envoi, nombre d'envois...).

Les fragments de messages sont stockés dans la hashmap \textrm{fragmentation\_map} \hyperref[sec:frag]{(voir section \ref{sec:frag})}. On utilise aussi une hashmap (\textrm{pmtu\_map}) pour faire correspondre un message au voisin dont on essaye d'estimer le pmtu.

Les messages à envoyer sont stockés dans une liste circulaire.

\subsection{Boucles principales}
\begin{itemize}
    \item Thread d'envoi
\begin{enumerate}
\item Gestion des voisins symétriques: s'ils n'ont pas envoyé un hello long dans les deux dernières minutes, ils deviennent potentiels, sinon si on ne leur a pas envoyé de hello long depuis 30 secondes on leur en envoie un.
\item Si le nombre de voisin symétrique est trop petit ; gestion des voisins potentiels: s'ils n'ont pas répondu à plus de 5 hello courts, ils sont retirés, sinon si on ne leur a pas envoyé un hello long depuis un certain temps (à croissance exponentielle), on leur en envoie un autre.
\item inondation des messages aux voisins, si un voisin n'a pas envoyé de ack après 5 envois, il est rétrogradé en potentiel.
\item envoi de tlv neighbour aux voisins pour lesquels on ne l'a pas fait depuis 2 minutes.
\item On retire un par un les messages de la file de messages et on les envoie.
\item suppression des vieux messages (ceux pour lesquels aucun data n'a été envoyé ou aucun ack reçu depuis plus de 45 secondes).
\item Suppression des vieux fragments de messages (ceux pour lesquels aucun data n'a été envoyé ou aucun ack reçu depuis plus de 60 secondes) \hyperref[sec:frag]{(voir section \ref{sec:frag})}.
\item appel à \textbf{pthread\_cond\_timedwait} qui bloque jusqu'à ce que la condition d'attente soit finie, ici c'est le cas quand on push un tlv \hyperref[sec:frag]{(voir section \ref{sec:agrega})}, ou après un certain temps en fonction des points ci-dessus (qui modifient une \textbf{struct timespec} pour indiquer la date de la prochaine action à effectuer).

\end{enumerate}
\item Thread de réception \\
  On boucle sur un select puis s'il y a un message reçu, on va lire la socket tant qu'il y a des messages à lire, et traiter les messages associés.
\item Thread de l'entrée standard \\
  On boucle sur un appel à \textbf{readline} puis s'il y a un message à lire, on le traite selon s'il s'agit d'une commande ou d'un message à envoyer.
\item Thread de l'interface web \\
  On boucle sur un select liée à la socket passive du serveur \textrm{HTTP} et au sockets \textrm{TCP} correspondantes aux connexions WebSocket. Dans le premier cas on répond immédiatement à la requête \textrm{HTTP} et dans le second cas on lit la trame websocket et on agit quand on en reçoit une avec le bit \textrm{FIN} à 1 \hyperref[sec:web]{(voir section \ref{sec:web})}.
\end{itemize}

\section{Conclusion}

Ce projet fut par bien des aspects le plus intéressant que nous ayons eu à réaliser en 3 années de licences d'informatique.

L'implémentation d'un chat décentralisé est une chose amusante en soit, d'autant plus que l'interface développée par Juliusz Chroboczek nous permettait de tester et communiquer en même temps avec tous les autres étudiants travaillant sur ce projet. De plus la mailing-list mise en place par le professeur nous maintenait éveillé aux proprets des autres groupes et nous amenait à échanger. Nous avons proposé un protocole d'échange de très gros messages, une association type de fichier/type de data et des TLV en vue de l'implémentation d'une sécurité cryptographique. On communique, il se passe des choses, le projet est vivant ; c'est pour ça qu'on fait du réseau et pas un shell\footnote{le projet de système du semestre 5}.\\

Nous avons aussi pu apprendre de nombreuses choses dépassant le contenu du cours. Autant sur des protocoles de programmation réseaux \hyperref[sec:web]{(voir section \ref{sec:web})} que sur des aspects de programmation C comme les threads et même les chaînes de contrôle Bash\hyperref[sec:con]{(voir section \ref{sec:con})}.\\

Le seul bémol est la programmation C ; nous avons eu de nombreux problèmes d'implémentation dans nos hashmap et hashset qui ne seraient jamais arrivés avec un langage disposant d'une bibliothèque standard plus conséquente. De plus, notre gestion de la concurrence est particulièrement lourde et probablement pas systématiquement optimale. L'implémentation de ce même projet dans un langage comme \textrm{Rust}\footnote{Que nous connaissons et aimons particulièrement.} ou \textrm{Go}\footnote{Que nous connaissons un peu et sommes désireux de connaître plus.} aurait été beaucoup plus simple et aurait conduit à une application vraisemblablement plus solide.

\end{document}

 % - introduction : projet fait par machin et bidule.
 %  - manuel
 %     - comment compiler
 %     - comment s'en servir
 %  - partie qui marche
 %     - sujet minimal
 %     - extensions
 %  - partie qui ne marche pas
 %     - qui ne marche pas parce qu'on est incompétents
 %     - qui ne marche pas parce qu'on a eu la flemme
 %  - parties pompées sur les copains (à qui merci)
 %  - détails d'implémentation
 %     - structure générale du programme
 %     - parties non-triviales que je vous explique gentiment
 %     - parties super bien faites dont on est fiers
 %     - parties mal faites dont on a un peu honte mais qu'on a la flemme
 %       d'améliorer
 %  - commentaires sur le protocole
 %     - telle partie du protocole est mal faite, voici comment l'améliorer
 %     - telle partie du protocole est difficile à implémenter, je vous déteste
 %  - conclusion
 %     - ce sujet nous a énormément apporté, il nous a ouvert l'esprit et
 %       a amélioré nos performances sexuelles.
