\documentclass[a4paper,10pt]{article} %type de document et paramètres


\usepackage{lmodern} %police de caractère
\usepackage[english,french]{babel} %package de langues
\usepackage[utf8]{inputenc} %package fondamental
\usepackage[T1]{fontenc} %package fondamental

\usepackage[top=3cm, bottom=3cm, left=3cm, right=3cm]{geometry} %permet de

\usepackage[pdftex, pdfauthor={Maxime Flin, Pierre Gimalac}, pdftitle={Rapport projet}, pdfsubject={chatgroup with udp flooding}, pdfkeywords={UDP, chatgroup, flooding}, colorlinks=true, linkcolor=black]{hyperref}

\author{Maxime Flin \& Pierre Gimalac}
\title{Rapport de Projet}

\begin{document}
\maketitle

\section{Compilation et exécution}
Le projet se compile grâce au makefile fourni, en faisant la commande \textit{'make all'}.\\

L'exécutable produit s'appelle \textit{chat}. Il possèdes des arguments optionnels qui sont dans l'ordre:
\begin{enumerate}
\item un numéro de port (un numéro aléatoire sera choisi si l'argument n'est pas fourni)
\item ``0'', ``1'', ``STDOUT'', ``STDERR'', ou un chemin vers un fichier (potentiellement à créer), indiquant où doivent être écris les logs. Par défaut les logs sont écris sur STDERR.
\item un pseudonyme (un pseudonyme aléatoire sera choisi dans une liste si l'argument n'est pas fourni).
\end{enumerate}

\section{Interface}
\subsection{Commandes}
L'interface possède un certain nombre de commandes, qui commencent par /:
\begin{description}
\item[add <addr> <port>] ajoute aux voisins potentiels les adresses associées à <addr> et <port>.
\item[name <name>] change le pseudonyme en <name>.
\item[random] change le pseudonyme de manière aléatoire.
\item[print] affiche la liste des voisins et voisins potentiels.
\item[juliusz] équivalent à ``/add jch.irif.fr 1212''
\item[neighbour] force l'envoi de tlv neighbour.
\item[clear] efface le terminal (fonctionne au moins sur bash et zsh).
\item[chid] change l'id de manière aléatoire.
\item[transfert <path>] envoie le fichier <path> avec le protocole de fragmentation.
\item[help] affiche l'ensemble des commandes.
\item[quit] quitte programme, envoie des tlv goaway au passage.
\end{description}

\subsection{Affichage}
Les messages affichés sont de différentes couleurs selon la sortie utilisées (STDOUT, STDERR, log) [les couleurs peuvent être facilement changées dans \textit{interface.h}].\\

La fonction \textit{cprint} dans \textit{utils} redéfinit \textit{dprintf} en mettant les messages à la bonne couleur selon la sortie voulue.

\section{Organisation du projet}
Le fichier \textit{main} contient la boucle principale du programme, une fonction d'initialisation et des fonctions de gestion de l'entrée standard.\\

Le sous-dossier \textbf{structs} contient les structures de données utilisées dans le projet: hashmap, hashset, liste chaînée, arraylist.\\

Le fichier \textit{interface} contient les fonctions liées à l'interface (changement de pseudonyme, gestion de commandes, affichages des messages).\\

Les fichiers \textit{network} et \textit{flooding} contiennent les fonctions liées à l'inondation, à la réception et à l'envoi de messages.\\

Les fichiers \textit{tlv}, \textit{checkers}, \textit{handlers}, \textit{onsend} contiennent des fonctions de gestion de tlv (création, vérification et gestion à la réception,...).\\

La plupart des structures utilisées pour l'implémentation du protocole sont définies dans \textit{types.h}.\\

Le fichier \textit{util} contient des fonctions diverses utilisées tout au long du projet.

\section{Sujet minimal}
Nous gérons les voisins avec plusieurs hashset: un pour les voisins potentiels et un pour les voisins symétriques.

Les messages reçus récemment et les messages à innonder sont chacun stockés par une hashmap. Dans le second cas, il s'agit d'une hashmap de hashset (chaque hashset contient les voisins encore à innonder pour le message donné).

Les fragments de messages sont stockés dans une hasmap. [voir extensions]

Les messages à envoyer sont stockés dans une liste circulaire.

La boucle principale fait dans l'ordre:
\begin{enumerate}
\item gestion des voisins symétriques: s'ils n'ont pas envoyé un hello long dans les deux dernières minutes, ils deviennent potentiels, sinon on leur envoie un hello long.
\item si le nombre de voisin symétrique n'est pas trop élevé, gestion des voisins potentiels: s'ils n'ont pas répondu à plus de 5 hellos courts, ils sont retirés, sinon on leur en envoie un autre.
\item innondation des messages aux voisins, si un voisin n'a pas envoyé de ack après 5 envois, il est rétrogradé en potentiel.
\item envoie de tlv neighbour aux voisins pour lesquels on ne l'a pas fait depuis 2 minutes.
\item on retire un par un les messages de la liste de messages et on les envoie.
\item suppression des vieux messages (ceux pour lesquels aucun data n'a été envoyé ou aucun ack reçu depuis plus de 45 secondes).
\item suppression des vieux fragments de messages (ceux pour lesquels aucun data n'a été envoyé ou aucun ack reçu depuis plus de 60 secondes) [voir extensions].
\item appel à \textit{select} sur STDIN et la socket utilisée, en donnant un timeout qui est le minimum des prochains moments où l'une des actions ci-dessus sera nécéssaire.
\item si STDIN est à lire, traitement de l'entrée standard (commande ou envoi de message).
\item s'il y a un message reçu, on va lire la socket un certains nombre de fois (pour éviter une congestion s'il y a beaucoup de messages reçus). Ce nombre est compris entre le nombre de voisins et deux fois le nombre de voisins et varie en fonction du nombre de messages reçus à chaque tour de boucle.
\end{enumerate}

\section{Extensions}
\subsection{Tlv 220}

% interface web
% tlv 221 ?
% crypto ?



\end{document}

 % - introduction : projet fait par machin et bidule.
 %  - manuel
 %     - comment compiler
 %     - comment s'en servir
 %  - partie qui marche
 %     - sujet minimal
 %     - extensions
 %  - partie qui ne marche pas
 %     - qui ne marche pas parce qu'on est incompétents
 %     - qui ne marche pas parce qu'on a eu la flemme
 %  - parties pompées sur les copains (à qui merci)
 %  - détails d'implémentation
 %     - structure générale du programme
 %     - parties non-triviales que je vous explique gentiment
 %     - parties super bien faites dont on est fiers
 %     - parties mal faites dont on a un peu honte mais qu'on a la flemme
 %       d'améliorer
 %  - commentaires sur le protocole
 %     - telle partie du protocole est mal faite, voici comment l'améliorer
 %     - telle partie du protocole est difficile à implémenter, je vous déteste
 %  - conclusion
 %     - ce sujet nous a énormément apporté, il nous a ouvert l'esprit et
 %       a amélioré nos performances sexuelles.
